\documentclass[12pt]{article}
\usepackage{fancyhdr}
\usepackage{amsmath, amsthm, amssymb}
\usepackage{graphicx}
\usepackage{hyperref}
%\usepackage{geometry}

\title{Com S 330 HW02}
\author{Siyu Lin \\Recitation Section 1\\ \texttt{siyul@iastate.edu}}
\begin{document}
\maketitle

\begin{enumerate}
\item Rosen, Section 1.4: Exercise 10 (a) (b) (d)
\item Rosen, Section 1.4: Exercise 14 (a) (d)
\item Rosen, Section 1.4: Exercise 24 (a) (c)
\item Rosen, Section 1.4: Exercise 46 (b)
\item Are $\forall x(P(x)\to Q(x))$and$\forall xP(x)\to Q(x)$logically equivalent?If yes, give a proof. If no, give a counterexample.
\item Rosen, Section 1.5: Exercise 10 (a) (d)
\item Rosen, Section 1.5: Exercise 12 (i) (n)
\item Rosen, Section 1.5: Exercise 36 (d)
\item Lehman et al. Problem 3.32
\item  Define predicates and prove the following using the appropriate rules of
inference:
\begin{enumerate}
    \item Beth, an ISU student, visited Brazil this summer. Everyone who visited Brazil this summer watched the World Cup. Therefore, an ISU student watched the World Cup.
    \item Some music majors are also computer science majors. Every music major can play the piano. There is a computer science major who can play the piano.
\end{enumerate}
\item Consider the following argument:
\begin{itshape}
Every computer science major takes discrete mathematics. Anyone who takes discrete mathematics understands logic. No one who understands logic will lose arguments. Therefore, no computer science majors will lose arguments.
\end{itshape}
\begin{enumerate}
\item Prove the argument using the rules of inference in Tables 1 and 2 of the book.
\item Prove the \textbf{universal transitivity} rule, which states that if $\forall x(P(x)\to Q(x))$ and $\forall x(P(x)\to R(x))$
\item Now, prove the previous argument using the \textbf{universal transitivity} rule.
\end{enumerate}
\item State whether the following arguments are correct. Explain your answer briefly.
\begin{enumerate}
\item All freshmen live in the dorms. Joe is not a freshman. Therefore, Joe does not live in the dorms.
\item  Ben likes all comedies. Ben likes ‘Hunger Games’. Therefore, ‘Hunger Games’ is a comedy.
\end{enumerate}

\end{enumerate}
\end{document}
