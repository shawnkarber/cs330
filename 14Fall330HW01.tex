\documentclass[12pt]{article}
\usepackage{fancyhdr}
\usepackage{amsmath, amsthm, amssymb}
\usepackage{graphicx}
\usepackage{hyperref}
%\usepackage{geometry}

\title{Com S 330 HW01}
\author{Siyu Lin \\ \texttt{siyul@iastate.edu}}
\begin{document}
\maketitle

1. Rosen, Section 1.1: Exercise 10 (b) (d) (h)

\indent Solution:\\
\indent $p$: The election is decided\\
\indent $q$: The votes have been counted\\
\indent (b) The election is decided or the votes have been counted.\\
\indent (d) The votes have counted implies that the election is decided.\\
\indent (h) The votes have not been counted, or the election is not decided and the votes have been counted.\\
\newline

2. Rosen, Section 1.1: Exercise 12 (c) (e)

\indent Solution:\\
\indent $p$: You have the flu.\\
\indent $q$: You miss the final exam.\\
\indent $r$: You pass the course.\\
\indent (c) You miss the final exam implies that you don’t pass the course.\\
\indent (e) You have the flu implies that you don’t pass the course, or you miss the exam implies that you don’t pass the course. \\
\newline

3. Rosen, Section 1.1: Exercise 24 (b) (h)

\indent Solution:\\
\indent (b) If you were born in the United States, you are to be a citizen of this country.\\
\indent (h) If you don’t begin your climb too late, you will reach the summit.\\
\newline

4. Rosen, Section 1.1: Exercise 28 (b)

\indent Solution:\\
\indent Converse: If I go to the beach, then it is a sunny summer day.\\
\indent Inverse: If it is not a sunny summer day, then I do not go to the beach.\\
\indent Contrapositive: If I do not go to the beach, then it is not a summer sunny day.\\
\newline

5. Rosen, Section 1.1: Exercise 34 (f)

\indent Solution:\\
\indent Truth Table:\\
\begin{center}
\begin{tabular}{c|c|c|c|c}
    $p$ & $q$ & $p \oplus q$ & $p \oplus \lnot q$ & $ (p \oplus q) \land (p \oplus \lnot q) $ \\
\hline
F & F & T & T & T \\
F & T & F & F & F \\
T & F & F & F & F \\
T & T & T & T & T \\
\end{tabular}
\end{center}

6. Rosen, Section 1.2: Exercise 18 

\indent Solution:\\
\begin{enumerate}
\item \label{Jas} $p$: Jasmine attends the party.
\item \label{Sam} $q$: Samir attends the party.
\item \label{Kan} $r$: Kanti attends the party.
\item \label{JasAndSam} $p \land q \to$ Jasmine will not be happy. If Jasmine and Samir attend the party, Jasmine will become unhappy.
\item \label{SamImpKan} $q \to r$: If Samir will attend, then Kanti will be there.
\item \label{NotJasNotKan} $\lnot p \to \lnot r$: If Jasmine does not attend, then Kanti will not attend.
\end{enumerate}

By \ref{JasAndSam} we have known that if \ref{Jas} and \ref{Sam} are both true, then Jasmine will not be happy. So if one of them is false, Jasmine will be happy. If \ref{Jas} is false, then from \ref{NotJasNotKan}, we know that \ref{Kan}  is false. So we can only invite Samir. But by \ref{SamImpKan}, since Kanti will not be there, Samir will not be there either, i.e, \ref{Sam}  is false. The result is none of three will attend. If Samir does not attend the party, i.e, \ref{Sam} is false, we can make \ref{Jas}  or \ref{Kan} true. By \ref{NotJasNotKan}, we know that if \ref{Kan} is true, then \ref{Jas} is true. And no one is unhappy. So we can invite Jasmine, Kanti or both of them.  
\newline

7. Rosen, Section 1.2: Exercise 24

\indent Solution:\\
\begin{enumerate}
\item \label{precon1} Only one of them is the knight, one of them is the knave and one of them is the spy.
\item \label{Cknave} $p \to \lnot r$: C is the knave.
\item \label{Aknight} $q \to p$: A is the knight.
\item \label{Cspy} $r \to ((p \land \lnot q) \lor (q \land \lnot p))$: C is the spy.
\item \label{Atrue} $p$: A tells the truth.
\item \label{Btrue} $q$: B tells the truth.
\item \label{Ctrue} $r$: C tells the truth.
\end{enumerate}

By \ref{precon1}, we know that at least of them tells the true, so we can assume that one of \ref{Cknave}, \ref{Aknight} and \ref{Cspy} is true.\\
\indent Assume that \ref{Cknave} is true, we know that \ref{Ctrue} is false, and A is either the knight or the spy and C is the knave. And by \ref{precon1}, we know that A is not the spy. Because if A is the spy, then B is the knight, \ref{Btrue} and \ref{Aknight} are true; B and A are knights contradicts with \ref{precon1}. The result is A is the knight, B is the spy and C is the knave.\\ 
\indent Assume that \ref{Aknight} is true, we know that A is the knight and C is the knave. Based on the \ref{precon1}, we can conclude that B is the spy. The result does not contradict with \ref{precon1}.\\
\indent Assume that \ref{Cspy} is true, we know that C is the spy. So \ref{Cknave} is false. By \ref{precon1}, we know that A is the knave and B is the knight, so \ref{Aknight} is true, which contradicts with the fact that A is the knave.\\
\indent So we know that A is the knight, B is the spy and C is the knave.
\newline

8. Rosen, Section 1.2: Exercise 34

\indent Solution:\\
\begin{enumerate}
\item \label{Kchat} $p$: Kevin is chatting.
\item \label{Hchat} $q$: Heather is chatting.
\item \label{Rchat} $r$: Randy is chatting.
\item \label{Vchat} $s$: Vijay is chatting.
\item \label{Achat} $t$: Abby is chatting.
\item \label{KorH} $p \lor q$
\item \label{RxorV} $(r \land \lnot s) \lor (\lnot r \land s)$
\item \label{AimpliesR} $t \to r$
\item \label{VandJ} $(p \land s) \lor (\lnot p \land \not s)$
\item \label{HimpAandK} $q \to t \land p$
\end{enumerate}

By \ref{AimpliesR}, we have $r \lor \lnot t$. By \ref{HimpAandK}, we have $\lnot q \lor (p \land t)$.\\
\indent By \ref{AimpliesR}, we know that $t$ is false or $r$ is true. \\ 
\indent So we assume that $t$ is false. Here we get $t \land p$ is false. By \ref{HimpAandK}, we know that $q$ is false. By \ref{KorH}, $p$ is true, so $\lnot p \land \lnot s$ is false. By \ref{VandJ}, we know that $s$ is true, so $r \land \lnot s$ is false. By \ref{RxorV}, $\lnot r$ is true. So we have Kevin is chatting, Heather is not chatting, Randy is not chatting, Vijay is chatting and Abby is not chatting.\\ 
\indent We can also assume that $r$ is true. The $\lnot r \land s$ is false. By \ref{RxorV}, $\lnot s$ is true. So $p \land s$ is false. By \ref{VandJ}, $\lnot p$ is true, so $p$ is false. As a result, $t \land p$ is false, and by \ref{HimpAandK}, $q$ is false. So $p$ and $q$ are false, which contradicts with \ref{KorH}. The assumption is wrong.\\
\indent According to the above inference, we have that Kevin is chatting, Heather is not chatting, Randy is not chatting, Vijay is chatting and Abby is not chatting.
\newline

9. 

\indent Solution:\\
\begin{align*}
N \to L \oplus N \equiv \lnot N \lor [L \oplus N] && \text{[Logical equivalences Involving Implications]}\\ 
\end{align*}
\begin{align*}
L \oplus N \to N \equiv \lnot [L \oplus N] \lor N &&  \text{[Logical equivalences Involving Implications]} \\
\end{align*}

\begin{center}
\begin{tabular}{c|c|c|c|c}
    $L$ & $N$ & $L \oplus N$ & $N \to L \oplus N$ & $L \oplus N \to N$ \\
\hline
T & T & F & F & T \\
T & F & T & T & F \\
F & T & T & T & T \\
F & F & F & T & T
\end{tabular}
\end{center}

By the truth table, we know that when $L$ is false and $N$ is true the direction out of Paradox is the North if and only if $L \oplus N$ is true. So $L$ is false, which means that Bill is not a liar.
\newline

10. Rosen, Section 1.3: Exercise 26 Use truth tables

\indent Solution:\\
\indent Truth table\\

\begin{center}
\begin{tabular}{c|c|c|c|c}
    $p$ & $q$ & $r$ & $\lnot p \to (q \to r)$ & $q \to (p \lor r)$ \\
\hline
T & T & T & T & T\\
T & T & F & T & T\\
T & F & T & T & T\\
T & F & F & T & T\\
F & T & T & T & T\\
F & T & F & F & F\\
F & F & T & T & T\\
F & F & F & T & T
\end{tabular}
\end{center}

11. 

\indent Solution:\\
\indent (a) Truth table\\
\begin{center}
\begin{tabular}{c|c|c|c|c}
    $p$ & $q$ & $r$ & $(p \to q) \land (p \to r)$ & $p \to (q \land r)$ \\
\hline
T & T & T & T & T \\
T & T & F & F & F \\
T & F & T & F & F \\
T & F & F & F & F \\
F & T & T & T & T \\
F & T & F & T & T \\
F & F & T & T & T \\
F & F & F & T & T
\end{tabular}
\end{center}

\indent (b) Deduction\\
\begin{align}
\label{pImpqAndpImpr}
[p \to q] \land [p \to r] &\equiv [\lnot p \lor q] \land [\lnot p \lor r]  && \text{[Logical equivalences Involving Implications]}\\
                          &\equiv \lnot p \lor [q \land r] && \text{[Distributive law]} 
\end{align}

\begin{align}
\label{pImpqAndr}
p \to [q \land r] \equiv \lnot p \lor [q\land r] && \text{[Logical equivalences Involving Implications]}
\end{align}

By \ref{pImpqAndpImpr} and \ref{pImpqAndr}, we know that they are logically equivalent.
\newline

12. 

\indent Solution:\\
\indent Truth table:\\
\begin{center}
\begin{tabular}{c|c|c|c}
    $p$ & $q$ & $p $ NOR $ q$ & $\lnot (p \lor q)$ \\
\hline
T & T & F & F\\
T & F & F & F\\
F & T & F & F\\
F & F & T & T
\end{tabular}
\end{center}

\indent We have known that AND, OR, and NOT are functionally complete. If we want to prove that NOR is functionally complete, we can express AND, OR, or NOT using NOR.\\

\begin{align}
\lnot p & \equiv \lnot [p \lor p] && \text{[Idempotent laws]}\\
        \label{NOTpEqpNORp}
        & \equiv p \text{ NOR } p && \text{[Definition of NOR]}
\end{align}

\begin{align}
p \lor q & \equiv \lnot p \text{ NOR } q && \text{[Definition of NOR]} \\
         & \equiv [p \text{ NOR } q] \text{[ NOR ]} [p \text{ NOR } q] && \text{By \ref{NOTpEqpNORp}}\\
        & \equiv p \text{ NOR } q && \text{[Definition of NOR]}
\end{align}
\begin{align}
p \land q & \equiv \lnot [\lnot p \lor \lnot q] && \text{[De Morgan's Law]} \\
          & \equiv \lnot p \text{ NOR } \lnot q && \text{[Definition of NOR]}\\
          & \equiv [p \text{ NOR } p] \text{ NOR } [q \text{ NOR } q] &&       \text{By \ref{NOTpEqpNORp}} 
\end{align}

\indent So we know how to express AND using NOR. Because AND is functionally complete, NOR is functionally complete as well. \\


\end{document}
