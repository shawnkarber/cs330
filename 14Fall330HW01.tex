\documentclass{article}
\usepackage{fancyhdr}
\usepackage{amsmath ,amsthm ,amssymb}
\usepackage{graphicx}
\usepackage{hyperref}

\title{Com S 330 HW01}
\author{Siyu Lin\\ \url{siyul@iastate.edu}}
\begin{document}
\maketitle

1. Rosen, Section 1.1: Exercise 10 (b) (d) (h)

\indent Solution:\\
\indent $p$: The election is decided\\
\indent $q$: The votes have been counted\\
\indent (b) The election is decided or the votes have been counted.\\
\indent (d) The votes have counted implies that the election is decided.\\
\indent (h) The votes have not been counted, or the election is not decided and the votes have been counted.\\
\newline

2. Rosen, Section 1.1: Exercise 12 (c) (e)

\indent Solution:\\
\indent $p$: You have the flu.\\
\indent $q$: You miss the final exam.\\
\indent $r$: You pass the course.\\
\indent (c) You miss the final exam implies that you don’t pass the course.\\
\indent (e) You have the flu implies that you don’t pass the course, or you miss the exam implies that you don’t pass the course. \\
\newline

3. Rosen, Section 1.1: Exercise 24 (b) (h)

\indent Solution:\\
\indent (b) If you were born in the United States, you are to be a citizen of this country.\\
\indent (h) If you don’t begin your climb too late, you will reach the summit.\\
\newline

4. Rosen, Section 1.1: Exercise 28 (b)

\indent Solution:\\
\indent Converse: If I go to the beach, then it is a sunny summer day.\\
\indent 1Inverse: If it is not a sunny summer day, then I don’t go to the beach.\\
\indent Contrapositive: If I don’t go to the beach, then it is not a summer sunny day.\\
\newline

5. Rosen, Section 1.1: Exercise 34 (f)

\indent Solution:\\
\indent Truth Table:\\
\begin{center}
\begin{tabular}{c|c|c|c|c}
    $p$ & $q$ & $p \lplus q$ & $p \lplus \lnot q$ & $ (p \plus q) \land (p \plus \lnot q) $ \\
\hline
F & F & T & T & T \\
F & T & F & F & F \\
T & F & F & F & F \\
T & T & T & T & T \\
\end{tabular}
\end{center}

6. Rosen, Section 1.2: Exercise 18 

Solution:
\begin{enumerate}
\item \label{Jas} $p$: Jasmine attends the party.
\item \label{Sam} $q$: Samir attends the party.
\item \label{Kan} $r$: Kanti attends the party.
\item \label{JasAndSam} $p \land q$ \to Jasmine will not be happy. If Jasmine and Samir attend the party, Jasmine will become unhappy.
\item \label{SamImpKan} $q \to r$: If Samir will attend, then Kanti will be there.
\item \label{NotJasNotKan} $\lnot p \lto \lnot r$: If Jasmine does not attend, then Kanti will not attend.
\end{enumerate}

By \ref{JasAndSam} we have known that if \ref{Jas} and \ref{Sam} are both true, then Jasmine will not be happy. So if one of them is false, Jasmine will be happy. If \ref{Jas} is false, then from \ref{NotJasNotKan}, we know that \ref{Kan}  is false. So we can only invite Samir. But by \ref{SamImpKan}, since Kanti will not be there, Samir will not be there either, i.e, \ref{Sam}  is false. The result is none of three will attend. If Samir does not attend the party, i.e, \ref{Sam} is false, we can make \ref{Jas}  or \ref{Kan} true. By \ref{NotJasNotKan}, we know that if \ref{Kan} is true, then \ref{Jas} is true. And no one is unhappy. So we can invite Jasmine, Kanti or both of them.  

\end{document}
